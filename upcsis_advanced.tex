\documentclass[supercite]{upcthesis}
\usepackage{lipsum}
\usepackage{makecell}
\usepackage{amsmath}
\usepackage{mathtools}
\usepackage{amsfonts,amssymb}
\usepackage{subfigure}
\usepackage{enumerate}
\usepackage{graphicx}
\usepackage{epstopdf}
\usepackage{etoolbox}
\usepackage{titlesec}
\usepackage{tocloft}

% \usepackage{ntheorem}
\usepackage{amsthm}





% \theorembodyfont{\songti}
% {
% \newtheorem{mdef}{\hskip 2em 定义}[section]  %按一级标题section编号
% \newtheorem{mthe}{\hskip 2em 定理}[section]
% \newtheorem{mlma}[mthe]{\hskip 2em 引理}[section]
% \newtheorem{mcor}[mthe]{\hskip 2em 推论}[section]
% \newtheorem{mprop}[mthe]{\hskip 2em 命题 }[section]
% \newtheorem*{mpro}{\hskip 2em 证明} %不编号
% \newtheorem*{msol}{\hskip 2em 解} %不编号
% \newtheorem{mexa}{\hskip 2em 例 }[section]
% }




% 中文定理环境
% \indent 为了段前空两格


% \theorembodyfont{\songti}
% {
\newtheorem{definition}{\indent 定义}[section]
\newtheorem{theorem}{\indent 定理}[section]
\newtheorem{lemma}[theorem]{\indent 引理}
\newtheorem{proposition}[theorem]{\indent 命题}
\newtheorem{corollary}[theorem]{\indent 推论}
\newtheorem{example}{\indent 例}[section]
\newtheorem{remark}{\indent 注}[section]
\newenvironment{solution}{\begin{proof}[\indent\textbf{解}]}{\end{proof}}
\renewcommand{\proofname}{\indent\bf 证明}
% }

% English theorem environment

% \theoremstyle{margin}
% {
% \newtheorem{theorem}{Theorem}[section]
% \newtheorem{lemma}[theorem]{Lemma}
% \newtheorem{proposition}[theorem]{Proposition}
% \newtheorem{corollary}[theorem]{Corollary}
% \newtheorem{definition}{Definition}[section]
% \newtheorem{remark}{Remark}[section]
% \newtheorem{example}{Example}[section]
% \newenvironment{solution}{\begin{proof}[Solution]}{\end{proof}}
% }


%%%%%%%%%调整subsection与subsubsection格式
\titlespacing*{\subsubsection}{0pt}{0.5ex plus .2ex minus .2ex}{%
	0.5ex plus .2ex
}
\titlespacing*{\subsection}{0pt}{0.5ex plus .2ex minus .2ex}{%
	0.5ex plus .2ex
}
%%%%%%%%%貌似是调整参考文献格式?
\let\oldthebibliography \thebibliography
\let\endoldthebibliography \endthebibliography
%\renewenvironment{thebibliography}[1]{%
%	\begin{oldthebibliography}{#1}%
%		\setlength{\parskip}{0ex}%
%		\setlength{\itemsep}{0ex}%
%		\setlength{\itemindent}{4ex}
%		\setlength{\leftmargin}{-3pt}%
%	}% 
%	{
%	\end{oldthebibliography}%
%}
%%%%%%%%%调整数学公式格式,貌似是设置公式编号格式?
\renewcommand{\theequation}{\thesection -\arabic{equation}}
\makeatletter
%%%%%%%%%貌似是调整参考文献文本格式(如缩进等),非参考文献自身格式
\renewenvironment{thebibliography}[1]
{\section*{\refname}%
	\@mkboth{\MakeUppercase\refname}{\MakeUppercase\refname}%
	\list{\@biblabel{\@arabic\c@enumiv}}%
	{\settowidth\labelwidth{\@biblabel{#1}}%
		\setlength{\itemindent}{\dimexpr\labelwidth+\labelsep}
		\leftmargin\z@
		\@openbib@code
		\usecounter{enumiv}%
		\let\p@enumiv\@empty
		\renewcommand\theenumiv{\@arabic\c@enumiv}}%
	\sloppy
	\clubpenalty4000
	\@clubpenalty \clubpenalty
	\widowpenalty4000%
	\sfcode`\.\@m}
{\def\@noitemerr
	{\@latex@warning{Empty `thebibliography' environment}}%
	\endlist}
%%%%%貌似是数学公式从每个章节(section)重新开始编号
\@addtoreset{equation}{section}
\makeatother

\title{线性表的设计和实现实现实现实现实现}

\author{张三}
\date{\today}
\supervisor{罗\hspace{0.53cm}翔}
\stuid{1401013101}
\classnum{电气工程及其自动化14-5班}
\subtitle{这是副标题}


%%%%%%%英文题目
\entitle{\begin{center}The design and implementation of the linear form\end{center}}
%\ensubtitle{This is EnSubTitle}
\begin{document}
\maketitle

\begin{cnabstract}{数据结构;面向对象;可视化;算法}
结构算法设计和演示(C++)树和查找是在面向对象思想和技术的指导下,采用面向对象的编程语言(C++)和面向对象的编程工具(Borland C++ Builder 6.0)开发出来的小型应用程序。它的功能主要是将数据结构中链表、栈、队列、树、查找、图和排序部分的典型算法和数据结构用面向对象的方法封装成类,并通过类的对外接口和对象之间的消息传递来实现这些算法,同时利用C++ Builder 6.0中丰富的控件资源和系统资源对算法实现过程的流程和特性加以动态的演示,从而起到在数据结构教学中帮助理解、辅助教学和自我学习的作用。
\end{cnabstract}
\begin{enabstract}{Write Criterion;Typeset Format;Graduation Project (Thesis)}
Abcdeafg Abcdefg Abcdefg Abcdefg Abcdefg Abcdefg Abcdefg Abcdefg Abcdefg Abcdefg Abcdefg Abcdefg Abcdefg Abcdeafg Abcdefg Abcdefg Abcdefg Abcdefg Abcdefg Abcdefg Abcdefg Abcdefg Abcdefg Abcdefg Abcdefg Abcdefg Abcdefg Abcdefg Abcdsefg Abcdefg Abcdefg Abcdefg Abcdefg Abcdefg Abcdefg Abcdefg Abcdefg Abcdefg Abcdefg Abcdefg Abcdefg Abcdefg Abcdefg Abcdefg Abcdefg Abcdefg Asbcdefg Abcdefg Abcdefg Abcdsefg Abcdefg Abcdefg Abcdaefg Abcdefg Abcdefg Abcdefg Abcdefg Abcdefg Abcdefg Abcdefg Abcdefg Abcdefg Abcdefg Abcdefg Abcdeafg Abcdefg Abcdefg Abcdefg Abcdefg Abceefg Abcdefg Abcdefg Abcdefg Abcdefg Abcdefg Abcdefg Abcdefg Abcdeefg Abcdefg Abcdefg Abcdefg Abcdefg Abcdefg Abcdefg Abcdefg Abcdefg Abcdefg Abdcdefg Abcdefg Abcdefg Abcdeafg Abcdedsdfg Abcdefg Abcdefg Abcdefg Abcdefg Abcdefg Abcdefg Abcdefg Abcdefg Abcdefg Abcdefg Abcdateafg Abcdsefg Abcdsefg Abcddefg Abcdefag Abcdedfg.
\end{enabstract}

\tableofcontents

% 加入示例的文章主要部分
\include{examplemain}
% %!TEX option = -shell-escape
\documentclass[supercite,fontset=windows]{upcthesis}

% \def\enEnvironment{} %取消此行注释获得英文环境支持,包括定理环境,交叉引用等
\def \twoSidePrint{} %取消此行注释以获得简单的双面打印效果,在特定位置添加空白页。
\def \eletronicDocument{} %取消此行注释以获得彩色高亮的超链接和交叉引用,便于编写时查看。打印时取消
%导言区设置
%%% 产生一段随机文本的宏包
\usepackage{lipsum}
\usepackage{zhlipsum}


\usepackage{makecell}
%%% 数学环境?
\usepackage{amsmath}
% \usepackage{mathtools}
\usepackage{amsfonts,amssymb}
\usepackage{unicode-math}

%%% 子图
\usepackage{subfigure}

%%% 列表环境
\usepackage{enumitem}
%%% 图片环境
\usepackage{graphicx}

%%%暂时没发现有什么用,注释掉
% \usepackage{epstopdf}
\usepackage{etoolbox}
\usepackage{titlesec}

% 算法/伪代码环境
\usepackage{algorithm}  
\usepackage{algorithmic} 
 % 代码环境
\usepackage{listings}

% \usepackage{tocloft}


%%% 令长表格自动换页的宏包
\usepackage{longtable}


\usepackage{subfiles}
%%%审阅功能
% \usepackage{changes}
%%%交叉引用宏包,要放在最后引入
\usepackage{hyperref}







%%%%%---------------------------------------------------------------------
%%% 定义定理环境的宏包
\usepackage{amsthm}
% 中英定理类环境的声明

% 定义定理样式
\newtheoremstyle{thesty}
{3pt} %环境前间距
{3pt} %环境后间距
{\songti} %定理内字体
{2em} %头部缩进
{\bfseries\songti} %定理头部字体
{} %头部后添加符号
{0.5em} %头部后间距
{} %theorem head spec

% 应用定理样式
\theoremstyle{thesty}

\ifx\enEnvironment\undefined
% 中文定理环境
{
	\newtheorem{definition}{定义}[section]
	\newtheorem{theorem}{定理}[section]
	\newtheorem{lemma}{引理}[section]
	\newtheorem{proposition}{命题}[section]
	\newtheorem{corollary}{推论}[section]
	\newtheorem{example}{例}[section]
	\newtheorem{remark}{注}[section]
}
\newenvironment{proofenv}[1]{\par\indent\songti\textbf{#1}\hspace{0.3em}}{}
\renewenvironment{proof}{\begin{proofenv}{证明:}}{\end{proofenv}\newline \rightline{\qedsymbol}\par}
\newenvironment{solution}{\begin{proofenv}{解:}}{\end{proofenv}}

\else

%% 英文定理环境
{
	\newtheorem{theorem}{Theorem}[section]
	\newtheorem{lemma}{Lemma}[section]
	\newtheorem{proposition}{Proposition}[section]
	\newtheorem{corollary}{Corollary}[section]
	\newtheorem{definition}{Definition}[section]
	\newtheorem{example}{Example}[section]
	\newtheorem{remark}{Remark}[section]
}

\newenvironment{proofenv}[1]{\par\indent\songti\textbf{#1}\hspace{0.3em}}{}
\renewenvironment{proof}{\begin{proofenv}{Proof:}}{\end{proofenv}\newline \rightline{\qedsymbol}\par}
\newenvironment{solution}{\begin{proofenv}{Solution:}}{\end{proofenv}}

\fi


%%%%%---------------------------------------------------------------------

% 交叉引用样式
\ifx \enEnvironment \undefined
%%%--------------------------中文环境引用名
\def\sectionautorefname~#1\null{%
	第~#1~节\null
}
\def\subsectionautorefname~#1\null{%
	第~#1~小节\null
}
\def\subsubsectionautorefname~#1\null{%
	第~#1~小节\null
}

\def\paragraphautorefname~#1\null{%
	段落~#1~\null
}
\def\subparagraphautorefname~#1\null{%
	段落~#1~\null
}

% 重新设置图表 auto ref
\def\figureautorefname~#1\null{%
	图~#1~\null
}
\def\tableautorefname~#1\null{%
	表~#1~\null
}

% 重新设置公式autoref
\def\equationautorefname~#1\null{%
	式~(#1)~\null
}


\def \definitionautorefname~#1\null{
	定义~#1~\null
}
\def \theoremautorefname~#1\null{
	定理~#1~\null
}
\def \corollaryautorefname~#1\null{
	推论~#1~\null
}
\def \lemmaautorefname~#1\null{
	引理~#1~\null
}
\def \propositionautorefname~#1\null{
	命题~#1~\null
}
\def \exampleautorefname~#1\null{
	例~#1~\null
}
\def \remarkautorefname~#1\null{
	注~#1~\null
}
\def \algorithmautorefname~#1\null{
	算法~#1~\null
}

\else

%%%--------------------------英文环境引用名

\def\sectionautorefname~#1\null{%
	Section~#1~\null
}
\def\subsectionautorefname~#1\null{%
	Subsection~#1~\null
}
\def\subsubsectionautorefname~#1\null{%
	Subsubsection~#1~\null
}

\def\paragraphautorefname~#1\null{%
	Paragraph~#1~\null
}
\def\subparagraphautorefname~#1\null{%
	Paragraph~#1~\null
}


% 重新设置图表 auto ref
\def\figureautorefname~#1\null{%
	Figure~#1~\null
}
\def\tableautorefname~#1\null{%
	Tbale~#1~\null
}

% 重新设置公式autoref
\def\equationautorefname~#1\null{%
	Equation~(#1)~\null
}



\def \definitionautorefname~#1\null{
	Definition~#1~\null
}
\def \theoremautorefname~#1\null{
	Theorem~#1~\null
}
\def \corollaryautorefname~#1\null{
	Corollary~#1~\null
}
\def \lemmaautorefname~#1\null{
	Lemma~#1~\null
}
\def \propositionautorefname~#1\null{
	Proposition~#1~\null
}
\def \exampleautorefname~#1\null{
	Example~#1~\null
}
\def \remarkautorefname~#1\null{
	Remark~#1~\null
}
\def \algorithmautorefname~#1\null{
	Algorithm~#1~\null
}


\captionsetup[figure]{name={Figure}}
\captionsetup[table]{name={Table}}
\fi











%%%%%---------------------------------------------------------------------

% 单双面打印控制

\ifx \twoSidePrint \undefined
    \def \ClearPageStyle{\clearpage}
\else
    \def \ClearPageStyle{
    		\clearpage
	    	\thispagestyle{empty}
	    	\quad
	    	\clearpage
    }
\fi

\newcounter{startpage}
\newcounter{endpage}
\setcounter{startpage}{1}
\setcounter{endpage}{1}

\newenvironment{PrintMode}{
	\clearpage
	\setcounter{startpage}{\value{page}}
}{	
	\clearpage
	\setcounter{endpage}{\value{page}}
	% \thestartpage
	% \theendpage
	\addtocounter{endpage}{\value{startpage}}

	\ifodd \value{endpage}
		\ClearPageStyle
		\ifx \twoSidePrint \undefined
		\else
			\addtocounter{page}{-1} % 如果环境内补了一页空白页,将实际页码减一
		\fi
	\fi
}

\newcommand \PrintModeSubfile[1]{\begin{PrintMode}\subfile{#1}\end{PrintMode}}



%%%%%---------------------------------------------------------------------


\ifx \eletronicDocument \undefined

\else
\hypersetup{
	% backref = page,
	% pagebackref = true,
	colorlinks = true,
	linkcolor = blue,
	citecolor = blue,
	urlcolor = blue,
	pdfborder = 000, %去掉链接红(黑)框
	% bookmarks = true,
	% bookmarksopen = true,
	bookmarksnumbered = true
}
\fi

%%%%%---------------------------------------------------------------------
% 代码设置
\lstset{
    % basicstyle          =   \sffamily,          % 基本代码风格
    % keywordstyle        =   \bfseries,          % 关键字风格
    commentstyle        =   \rmfamily\itshape,  % 注释的风格,斜体
    % stringstyle         =   \ttfamily,  % 字符串风格
    flexiblecolumns,                % 别问为什么,加上这个
    numbers             =   left,   % 行号的位置在左边
    showspaces          =   false,  % 是否显示空格,显示了有点乱,所以不现实了
    numberstyle         =   \zihao{-5}\ttfamily,    % 行号的样式,小五号,tt等宽字体
    showstringspaces    =   false,
    captionpos          =   t,      % 这段代码的名字所呈现的位置,t指的是top上面
    % frame               =   lrtb,   % 显示边框
     breaklines      =   true,   % 自动换行,建议不要写太长的行
    columns         =   fixed,  % 强制等宽字体,更美观
}


% 列表宏包样式
\setlist{noitemsep}

\setmathfont{Cambria Math}


% 表格设定
% 添加复杂的表格宏
\usepackage{booktabs}
% 设置三线表格式
\setlength{\heavyrulewidth}{1.5pt}
\setlength{\lightrulewidth}{0.5pt}
\setlength{\cmidrulewidth}{0.5pt}
\setlength{\aboverulesep}{0pt}
\setlength{\belowrulesep}{0pt}
\setlength{\abovetopsep}{0pt}
\setlength{\belowbottomsep}{0pt}




% 封面各项参数
\makeatother
\title{线性表的设计和实现}
% \subtitle{这是副标题}
% \ensubtitle{This is EnSubTitle}
\entitle{The design and implementation of the linear form}
\author{张\hspace{1em}三}
\date{\today}
\supervisor{李\hspace{1em}四}
\stuid{1401013101}
\classnum{电气工程及其自动化14-5班}

\usepackage{tabu}
\begin{document}
%%%%%---------------------------------------------------------------------
% 封面页
\maketitle
\ClearPageStyle
% 中文摘要
\PrintModeInsert{\subfile{sections/before/cnabstract}}
% 英文摘要
\PrintModeInsert{\subfile{sections/before/enabstract}}
% 目录页
\PrintModeInsert{\tableofcontents}
% 设置页码
\setcounter{page}{1}

%%%%%---------------------------------------------------------------------
%%% 你的章节,使用时根据你的需要来导入,建议放在 sections/body/ 文件夹内,并且规范命名。

% \PrintModeInsert{\subfile{sections/body/section:引言.tex}}
% \PrintModeInsert{\subfile{sections/body/section:章节命名.tex}}
% ......

%%%%%---------------------------------------------------------------------

% 重设置页眉样式
\pagestyle{afterbody}
% 致谢页
\PrintModeInsert{\subfile{sections/after/thankpage}}
% 参考文献页
\PrintModeInsert{\subfile{sections/after/reference}}
%附录,不需要的同学可以注释此部分
\PrintModeInsert{\subfile{sections/after/appendix}}
\end{document}




\begin{thankpage}
        大学四年的学习生活即将结束,在此,我要感谢所有曾经教导过我的老师和关心过我的同学,他们在我成长过程中给予了我很大的帮助。本文能够成功的完成,要特别感谢我的导师XXX教授的关怀和教导。

        ……
\end{thankpage}





%\bibliography{./bibs/bibliography.bib}

%%%%%之前的.bib生成引用的样式与word要求不一致。能力有限,暂时无法修改
%%%%%样式文件不同于学校的要求。劳烦同学们手动引用文献

%%%%%手动指定在目录添加参考文献条目
\clearpage
\pagestyle{afterbody}
\phantomsection
\addcontentsline{toc}{section}{参考文献}

% 按照学校word模板中对参考文献的要求,列出以下几点给同学们参考


% 列出的参考文献必须在正文中有引用,并且需按正文中出现的次序进行排序。同一文献出现多次,只用同一标号

% 参考文献里的标点符号均为英文格式输入,每个标点符号与后面的内容之间要空一格。参考文献的各项条目使用逗号分割,最后要有句点。


% 参考文献应不少于10篇(外文文献至少2篇,外语专业应以外文文献为主)。

% 文献引用的格式大致为:

% 作者1, 作者2, 作者3, et. al, 题目, 期刊, 时间, 期数(卷数), 起-始页,网址.

% 其中不那么重要的或没有的部分可不写
% 其中:

% 英文作者,名缩写(老外是名在前,姓在后),如:Robert Jort缩写为:R. Jort,名字两个单词的,G. H. Golub。作者太多不适合全部列出的,写上 et. al,

% 英文题目除专有名词外,仅第一个单词首字母大写
% 题目中表示文献类型的符号:[M] [J] 等一律删掉,不允许出现。
% 期刊名应写全称,不知道的可以上网搜索。英文期刊中实词首字母的写。
% 一些英文常见期刊:

% 日期统一改为如下格式 2003.5.12


\begin{thebibliography}{99}
\bibitem{1} 严蔚敏, 吴伟民, 数据结构, 北京: 清华大学出版社, 1997.4.
\bibitem{2} 沈晴霓, 聂青, 苏京霞, 现代程序设计—C++与数据结构面向对象的方法与实现, 北京: 北京理工大学出版社, 2002.8.
\bibitem{3} T. Connolly, C. Begg, Database systems, 北京: 电子科技工业出版社, 2004.7.
\bibitem{4} R. Bate, S. Shrum, CMM Integration framework, CMU/SEI Spotlight, 1998, 4(3): 25-28.
\bibitem{5} J.P. Kuilboer, N. Ashrafi, Software process and product improvement, Physical Review A, 2000, 42(1): 27-34.
\bibitem{6} 张美金, 吴大伟, 基于ASP技术的远程教育系统体系结构的研究, http://172.50.0.88:86 /~cddbn/Y517807/pdf/index.htm, 2003-05-01.
\bibitem{7} 王伟国, 刘永萍, 王生年等, B/S模式网上考试系统分析与设计, 石河子大学学报(自然科学版), 2003, 6(2): 145-147.
\bibitem{8} …
\bibitem{9} …
\bibitem{10} …
\end{thebibliography}




%%%%%%%%附录,不需要的同学可以删去此部分

% 子文件在单独编译时会报
% “LaTeX Warning: You have requested document class `../../upcthesis', but the document class provides `upcthesis'.”
% 的警告,其实并没有什么影响,忽略即可
\documentclass[supercite,fontset=windows]{../../upcthesis}
\ifx\multiFile\undefined %防止在主文件中进行编译时多次导入导言区设置文件
%%% 产生一段随机文本的宏包
\usepackage{lipsum}
\usepackage{zhlipsum}


\usepackage{makecell}
%%% 数学环境?
\usepackage{amsmath}
% \usepackage{mathtools}
\usepackage{amsfonts,amssymb}
\usepackage{unicode-math}

%%% 子图
\usepackage{subfigure}

%%% 列表环境
\usepackage{enumitem}
%%% 图片环境
\usepackage{graphicx}

%%%暂时没发现有什么用,注释掉
% \usepackage{epstopdf}
\usepackage{etoolbox}
\usepackage{titlesec}

% 算法/伪代码环境
\usepackage{algorithm}  
\usepackage{algorithmic} 
 % 代码环境
\usepackage{listings}

% \usepackage{tocloft}


%%% 令长表格自动换页的宏包
\usepackage{longtable}


\usepackage{subfiles}
%%%审阅功能
% \usepackage{changes}
%%%交叉引用宏包,要放在最后引入
\usepackage{hyperref}







%%%%%---------------------------------------------------------------------
%%% 定义定理环境的宏包
\usepackage{amsthm}
% 中英定理类环境的声明

% 定义定理样式
\newtheoremstyle{thesty}
{3pt} %环境前间距
{3pt} %环境后间距
{\songti} %定理内字体
{2em} %头部缩进
{\bfseries\songti} %定理头部字体
{} %头部后添加符号
{0.5em} %头部后间距
{} %theorem head spec

% 应用定理样式
\theoremstyle{thesty}

\ifx\enEnvironment\undefined
% 中文定理环境
{
	\newtheorem{definition}{定义}[section]
	\newtheorem{theorem}{定理}[section]
	\newtheorem{lemma}{引理}[section]
	\newtheorem{proposition}{命题}[section]
	\newtheorem{corollary}{推论}[section]
	\newtheorem{example}{例}[section]
	\newtheorem{remark}{注}[section]
}
\newenvironment{proofenv}[1]{\par\indent\songti\textbf{#1}\hspace{0.3em}}{}
\renewenvironment{proof}{\begin{proofenv}{证明:}}{\end{proofenv}\newline \rightline{\qedsymbol}\par}
\newenvironment{solution}{\begin{proofenv}{解:}}{\end{proofenv}}

\else

%% 英文定理环境
{
	\newtheorem{theorem}{Theorem}[section]
	\newtheorem{lemma}{Lemma}[section]
	\newtheorem{proposition}{Proposition}[section]
	\newtheorem{corollary}{Corollary}[section]
	\newtheorem{definition}{Definition}[section]
	\newtheorem{example}{Example}[section]
	\newtheorem{remark}{Remark}[section]
}

\newenvironment{proofenv}[1]{\par\indent\songti\textbf{#1}\hspace{0.3em}}{}
\renewenvironment{proof}{\begin{proofenv}{Proof:}}{\end{proofenv}\newline \rightline{\qedsymbol}\par}
\newenvironment{solution}{\begin{proofenv}{Solution:}}{\end{proofenv}}

\fi


%%%%%---------------------------------------------------------------------

% 交叉引用样式
\ifx \enEnvironment \undefined
%%%--------------------------中文环境引用名
\def\sectionautorefname~#1\null{%
	第~#1~节\null
}
\def\subsectionautorefname~#1\null{%
	第~#1~小节\null
}
\def\subsubsectionautorefname~#1\null{%
	第~#1~小节\null
}

\def\paragraphautorefname~#1\null{%
	段落~#1~\null
}
\def\subparagraphautorefname~#1\null{%
	段落~#1~\null
}

% 重新设置图表 auto ref
\def\figureautorefname~#1\null{%
	图~#1~\null
}
\def\tableautorefname~#1\null{%
	表~#1~\null
}

% 重新设置公式autoref
\def\equationautorefname~#1\null{%
	式~(#1)~\null
}


\def \definitionautorefname~#1\null{
	定义~#1~\null
}
\def \theoremautorefname~#1\null{
	定理~#1~\null
}
\def \corollaryautorefname~#1\null{
	推论~#1~\null
}
\def \lemmaautorefname~#1\null{
	引理~#1~\null
}
\def \propositionautorefname~#1\null{
	命题~#1~\null
}
\def \exampleautorefname~#1\null{
	例~#1~\null
}
\def \remarkautorefname~#1\null{
	注~#1~\null
}
\def \algorithmautorefname~#1\null{
	算法~#1~\null
}

\else

%%%--------------------------英文环境引用名

\def\sectionautorefname~#1\null{%
	Section~#1~\null
}
\def\subsectionautorefname~#1\null{%
	Subsection~#1~\null
}
\def\subsubsectionautorefname~#1\null{%
	Subsubsection~#1~\null
}

\def\paragraphautorefname~#1\null{%
	Paragraph~#1~\null
}
\def\subparagraphautorefname~#1\null{%
	Paragraph~#1~\null
}


% 重新设置图表 auto ref
\def\figureautorefname~#1\null{%
	Figure~#1~\null
}
\def\tableautorefname~#1\null{%
	Tbale~#1~\null
}

% 重新设置公式autoref
\def\equationautorefname~#1\null{%
	Equation~(#1)~\null
}



\def \definitionautorefname~#1\null{
	Definition~#1~\null
}
\def \theoremautorefname~#1\null{
	Theorem~#1~\null
}
\def \corollaryautorefname~#1\null{
	Corollary~#1~\null
}
\def \lemmaautorefname~#1\null{
	Lemma~#1~\null
}
\def \propositionautorefname~#1\null{
	Proposition~#1~\null
}
\def \exampleautorefname~#1\null{
	Example~#1~\null
}
\def \remarkautorefname~#1\null{
	Remark~#1~\null
}
\def \algorithmautorefname~#1\null{
	Algorithm~#1~\null
}


\captionsetup[figure]{name={Figure}}
\captionsetup[table]{name={Table}}
\fi











%%%%%---------------------------------------------------------------------

% 单双面打印控制

\ifx \twoSidePrint \undefined
    \def \ClearPageStyle{\clearpage}
\else
    \def \ClearPageStyle{
    		\clearpage
	    	\thispagestyle{empty}
	    	\quad
	    	\clearpage
    }
\fi

\newcounter{startpage}
\newcounter{endpage}
\setcounter{startpage}{1}
\setcounter{endpage}{1}

\newenvironment{PrintMode}{
	\clearpage
	\setcounter{startpage}{\value{page}}
}{	
	\clearpage
	\setcounter{endpage}{\value{page}}
	% \thestartpage
	% \theendpage
	\addtocounter{endpage}{\value{startpage}}

	\ifodd \value{endpage}
		\ClearPageStyle
		\ifx \twoSidePrint \undefined
		\else
			\addtocounter{page}{-1} % 如果环境内补了一页空白页,将实际页码减一
		\fi
	\fi
}

\newcommand \PrintModeSubfile[1]{\begin{PrintMode}\subfile{#1}\end{PrintMode}}



%%%%%---------------------------------------------------------------------


\ifx \eletronicDocument \undefined

\else
\hypersetup{
	% backref = page,
	% pagebackref = true,
	colorlinks = true,
	linkcolor = blue,
	citecolor = blue,
	urlcolor = blue,
	pdfborder = 000, %去掉链接红(黑)框
	% bookmarks = true,
	% bookmarksopen = true,
	bookmarksnumbered = true
}
\fi

%%%%%---------------------------------------------------------------------
% 代码设置
\lstset{
    % basicstyle          =   \sffamily,          % 基本代码风格
    % keywordstyle        =   \bfseries,          % 关键字风格
    commentstyle        =   \rmfamily\itshape,  % 注释的风格,斜体
    % stringstyle         =   \ttfamily,  % 字符串风格
    flexiblecolumns,                % 别问为什么,加上这个
    numbers             =   left,   % 行号的位置在左边
    showspaces          =   false,  % 是否显示空格,显示了有点乱,所以不现实了
    numberstyle         =   \zihao{-5}\ttfamily,    % 行号的样式,小五号,tt等宽字体
    showstringspaces    =   false,
    captionpos          =   t,      % 这段代码的名字所呈现的位置,t指的是top上面
    % frame               =   lrtb,   % 显示边框
     breaklines      =   true,   % 自动换行,建议不要写太长的行
    columns         =   fixed,  % 强制等宽字体,更美观
}


% 列表宏包样式
\setlist{noitemsep}

\setmathfont{Cambria Math}


% 表格设定
% 添加复杂的表格宏
\usepackage{booktabs}
% 设置三线表格式
\setlength{\heavyrulewidth}{1.5pt}
\setlength{\lightrulewidth}{0.5pt}
\setlength{\cmidrulewidth}{0.5pt}
\setlength{\aboverulesep}{0pt}
\setlength{\belowrulesep}{0pt}
\setlength{\abovetopsep}{0pt}
\setlength{\belowbottomsep}{0pt}

\fi
% 导入这两个文件时使用的是相对路径,所以不要胡乱修改目录结构,否则会造成模板无法使用

\begin{document}


\begin{generalappendices}
        \section{名词术语及缩略词}
        \subsection{Some Appendix}
        \lipsum[11]
        \section{Appendix 2}
        \subsection{Some Other Appendix}
\end{generalappendices}





\end{document}


\end{document}