%%% 产生一段随机文本的宏包
\usepackage{lipsum}
\usepackage{zhlipsum}


\usepackage{makecell}
%%% 数学环境?
\usepackage{amsmath}
% \usepackage{mathtools}
\usepackage{amsfonts,amssymb}
\usepackage{unicode-math}

%%% 子图
\usepackage{subfigure}

%%% 列表环境
\usepackage{enumitem}
%%% 图片环境
\usepackage{graphicx}

%%%暂时没发现有什么用,注释掉
% \usepackage{epstopdf}
\usepackage{etoolbox}
\usepackage{titlesec}

% 算法/伪代码环境
\usepackage{algorithm}  
\usepackage{algorithmic} 
 % 代码环境
\usepackage{listings}

% \usepackage{tocloft}


%%% 令长表格自动换页的宏包
\usepackage{longtable}


\usepackage{subfiles}
%%%审阅功能
% \usepackage{changes}
%%%交叉引用宏包,要放在最后引入
\usepackage{hyperref}







%%%%%---------------------------------------------------------------------
%%% 定义定理环境的宏包
\usepackage{amsthm}
% 中英定理类环境的声明

% 定义定理样式
\newtheoremstyle{thesty}
{3pt} %环境前间距
{3pt} %环境后间距
{\songti} %定理内字体
{2em} %头部缩进
{\bfseries\songti} %定理头部字体
{} %头部后添加符号
{0.5em} %头部后间距
{} %theorem head spec

% 应用定理样式
\theoremstyle{thesty}

\ifx\enEnvironment\undefined
% 中文定理环境
{
	\newtheorem{definition}{定义}[section]
	\newtheorem{theorem}{定理}[section]
	\newtheorem{lemma}{引理}[section]
	\newtheorem{proposition}{命题}[section]
	\newtheorem{corollary}{推论}[section]
	\newtheorem{example}{例}[section]
	\newtheorem{remark}{注}[section]
}
\newenvironment{proofenv}[1]{\par\indent\songti\textbf{#1}\hspace{0.3em}}{}
\renewenvironment{proof}{\begin{proofenv}{证明:}}{\end{proofenv}\newline \rightline{\qedsymbol}\par}
\newenvironment{solution}{\begin{proofenv}{解:}}{\end{proofenv}}

\else

%% 英文定理环境
{
	\newtheorem{theorem}{Theorem}[section]
	\newtheorem{lemma}{Lemma}[section]
	\newtheorem{proposition}{Proposition}[section]
	\newtheorem{corollary}{Corollary}[section]
	\newtheorem{definition}{Definition}[section]
	\newtheorem{example}{Example}[section]
	\newtheorem{remark}{Remark}[section]
}

\newenvironment{proofenv}[1]{\par\indent\songti\textbf{#1}\hspace{0.3em}}{}
\renewenvironment{proof}{\begin{proofenv}{Proof:}}{\end{proofenv}\newline \rightline{\qedsymbol}\par}
\newenvironment{solution}{\begin{proofenv}{Solution:}}{\end{proofenv}}

\fi


%%%%%---------------------------------------------------------------------

% 交叉引用样式
\ifx \enEnvironment \undefined
%%%--------------------------中文环境引用名
\def\sectionautorefname~#1\null{%
	第~#1~节\null
}
\def\subsectionautorefname~#1\null{%
	第~#1~小节\null
}
\def\subsubsectionautorefname~#1\null{%
	第~#1~小节\null
}

\def\paragraphautorefname~#1\null{%
	段落~#1~\null
}
\def\subparagraphautorefname~#1\null{%
	段落~#1~\null
}

% 重新设置图表 auto ref
\def\figureautorefname~#1\null{%
	图~#1~\null
}
\def\tableautorefname~#1\null{%
	表~#1~\null
}

% 重新设置公式autoref
\def\equationautorefname~#1\null{%
	式~(#1)~\null
}


\def \definitionautorefname~#1\null{
	定义~#1~\null
}
\def \theoremautorefname~#1\null{
	定理~#1~\null
}
\def \corollaryautorefname~#1\null{
	推论~#1~\null
}
\def \lemmaautorefname~#1\null{
	引理~#1~\null
}
\def \propositionautorefname~#1\null{
	命题~#1~\null
}
\def \exampleautorefname~#1\null{
	例~#1~\null
}
\def \remarkautorefname~#1\null{
	注~#1~\null
}
\def \algorithmautorefname~#1\null{
	算法~#1~\null
}

\else

%%%--------------------------英文环境引用名

\def\sectionautorefname~#1\null{%
	Section~#1~\null
}
\def\subsectionautorefname~#1\null{%
	Subsection~#1~\null
}
\def\subsubsectionautorefname~#1\null{%
	Subsubsection~#1~\null
}

\def\paragraphautorefname~#1\null{%
	Paragraph~#1~\null
}
\def\subparagraphautorefname~#1\null{%
	Paragraph~#1~\null
}


% 重新设置图表 auto ref
\def\figureautorefname~#1\null{%
	Figure~#1~\null
}
\def\tableautorefname~#1\null{%
	Tbale~#1~\null
}

% 重新设置公式autoref
\def\equationautorefname~#1\null{%
	Equation~(#1)~\null
}



\def \definitionautorefname~#1\null{
	Definition~#1~\null
}
\def \theoremautorefname~#1\null{
	Theorem~#1~\null
}
\def \corollaryautorefname~#1\null{
	Corollary~#1~\null
}
\def \lemmaautorefname~#1\null{
	Lemma~#1~\null
}
\def \propositionautorefname~#1\null{
	Proposition~#1~\null
}
\def \exampleautorefname~#1\null{
	Example~#1~\null
}
\def \remarkautorefname~#1\null{
	Remark~#1~\null
}
\def \algorithmautorefname~#1\null{
	Algorithm~#1~\null
}


\captionsetup[figure]{name={Figure}}
\captionsetup[table]{name={Table}}
\fi











%%%%%---------------------------------------------------------------------

% 单双面打印控制

\ifx \twoSidePrint \undefined
    \def \ClearPageStyle{\clearpage}
\else
    \def \ClearPageStyle{
    		\clearpage
	    	\thispagestyle{empty}
	    	\quad
	    	\clearpage
    }
\fi

\newcounter{startpage}
\newcounter{endpage}
\setcounter{startpage}{1}
\setcounter{endpage}{1}

\newenvironment{PrintMode}{
	\clearpage
	\setcounter{startpage}{\value{page}}
}{	
	\clearpage
	\setcounter{endpage}{\value{page}}
	% \thestartpage
	% \theendpage
	\addtocounter{endpage}{\value{startpage}}

	\ifodd \value{endpage}
		\ClearPageStyle
		\ifx \twoSidePrint \undefined
		\else
			\addtocounter{page}{-1} % 如果环境内补了一页空白页,将实际页码减一
		\fi
	\fi
}

\newcommand \PrintModeSubfile[1]{\begin{PrintMode}\subfile{#1}\end{PrintMode}}



%%%%%---------------------------------------------------------------------


\ifx \eletronicDocument \undefined

\else
\hypersetup{
	% backref = page,
	% pagebackref = true,
	colorlinks = true,
	linkcolor = blue,
	citecolor = blue,
	urlcolor = blue,
	pdfborder = 000, %去掉链接红(黑)框
	% bookmarks = true,
	% bookmarksopen = true,
	bookmarksnumbered = true
}
\fi

%%%%%---------------------------------------------------------------------
% 代码设置
\lstset{
    % basicstyle          =   \sffamily,          % 基本代码风格
    % keywordstyle        =   \bfseries,          % 关键字风格
    commentstyle        =   \rmfamily\itshape,  % 注释的风格,斜体
    % stringstyle         =   \ttfamily,  % 字符串风格
    flexiblecolumns,                % 别问为什么,加上这个
    numbers             =   left,   % 行号的位置在左边
    showspaces          =   false,  % 是否显示空格,显示了有点乱,所以不现实了
    numberstyle         =   \zihao{-5}\ttfamily,    % 行号的样式,小五号,tt等宽字体
    showstringspaces    =   false,
    captionpos          =   t,      % 这段代码的名字所呈现的位置,t指的是top上面
    % frame               =   lrtb,   % 显示边框
     breaklines      =   true,   % 自动换行,建议不要写太长的行
    columns         =   fixed,  % 强制等宽字体,更美观
}


% 列表宏包样式
\setlist{noitemsep}

\setmathfont{Cambria Math}


% 表格设定
% 添加复杂的表格宏
\usepackage{booktabs}
% 设置三线表格式
\setlength{\heavyrulewidth}{1.5pt}
\setlength{\lightrulewidth}{0.5pt}
\setlength{\cmidrulewidth}{0.5pt}
\setlength{\aboverulesep}{0pt}
\setlength{\belowrulesep}{0pt}
\setlength{\abovetopsep}{0pt}
\setlength{\belowbottomsep}{0pt}
